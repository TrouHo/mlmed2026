% Report.3.tex  
\documentclass[conference]{IEEEtran}

\usepackage{graphicx}
\usepackage{booktabs}
\usepackage{amsmath}
\usepackage{hyperref}

\title{COVID-19 Infection Segmentation from Chest X-ray Images using a Lightweight U-Net}

\author{
\IEEEauthorblockN{Ho Huyen Chau}
\IEEEauthorblockA{
Machine Learning in Medicine 2026\\
University of Science and Technology of Hanoi (USTH)\\
Email: chauhh.23bi14067@usth.edu.vn
}
}

\begin{document}
\maketitle

\begin{abstract}
This report presents a practical study on segmenting COVID-19 infection regions in chest X-ray (CXR) images using the COVID-QU-Ex dataset. I implement a lightweight U-Net model for binary infection segmentation and evaluate performance using the Dice coefficient. Due to computational constraints, I train on a reduced subset of the dataset while demonstrating a complete pipeline. 

\end{abstract}

\begin{IEEEkeywords}
COVID-19, chest X-ray, medical image segmentation, U-Net, Dice coefficient
\end{IEEEkeywords}

% -------------------------
\section{Introduction}
Chest X-ray imaging is widely used for diagnosing lung diseases due to its low cost and availability. For COVID-19, localizing infected regions can provide pixel-level information beyond image-level classification. In this report, I study infection segmentation on chest X-ray images using the COVID-QU-Ex dataset and implement a lightweight U-Net model to predict binary infection masks. The goal is to demonstrate an end-to-end segmentation workflow under limited computational resources, rather than achieving state-of-the-art performance.

% -------------------------
\section{Dataset Description}
\subsection{Dataset Overview}
The COVID-QU-Ex dataset contains chest X-ray (CXR) images grouped into three categories: COVID-19, Non-COVID infections (e.g., viral or bacterial pneumonia), and Normal cases. In addition to the images, the dataset provides ground-truth segmentation masks for lung regions and infection regions, enabling supervised segmentation experiments.

\subsection{Folder Structure}
The dataset is organized into predefined splits \texttt{Train}, \texttt{Val}, and \texttt{Test}. For each class, the following folders are provided:
\begin{itemize}
  \item \texttt{images/}: chest X-ray images
  \item \texttt{infection masks/}: infection segmentation masks
  \item \texttt{lung masks/}: lung segmentation masks
\end{itemize}

\subsection{Dataset Imbalance}
Table~\ref{tab:class_dist} reports the number of training images per class based on my dataset inspection. COVID-19 has a larger number of samples compared to Non-COVID and Normal cases, indicating a class imbalance.

\begin{table}[ht]
\centering
\caption{Training set class distribution (COVID-QU-Ex).}
\label{tab:class_dist}
\begin{tabular}{l r}
\toprule
\textbf{Class} & \textbf{\# Images (Train)} \\
\midrule
COVID-19 & 1864 \\
Non-COVID & 932 \\
Normal & 932 \\
\bottomrule
\end{tabular}
\end{table}


\subsection{Empty Infection Masks}
When scanning infection masks in the training set, I found that 1864 out of 3728 masks are empty (all-zero). This strongly suggests that many samples (especially Normal and often Non-COVID) contain no annotated infection regions. Table~\ref{tab:empty_masks} summarizes this observation.

\begin{table}[ht]
\centering
\caption{Empty infection masks in the training split.}
\label{tab:empty_masks}
\begin{tabular}{l r}
\toprule
\textbf{Property} & \textbf{Value} \\
\midrule
Total infection masks (Train) & 3728 \\
Empty infection masks (Train) & 1864 \\
Empty mask ratio & 0.50 \\
\bottomrule
\end{tabular}
\end{table}

Therefore, I decided to train the model for two different circumstances: COVID-19 only and full three classes included. 

% -------------------------
\section{Methodology}
\subsection{Preprocessing}
All images are loaded as grayscale and resized to a fixed resolution to reduce computation. In my main experiment, I set the input size to $256 \times 256$. Pixel intensities are normalized to $[0,1]$. Infection masks are resized using nearest-neighbor interpolation and binarized to ensure consistent labels. During training, I apply light augmentation (random horizontal flip and small random rotations) to improve robustness.

\subsection{Model Architecture}
I use a lightweight U-Net architecture for binary segmentation. U-Net follows an encoder--decoder design with skip connections to combine spatial details from early layers with semantic features in deeper layers. The model outputs a single-channel logit map which is converted to probabilities using a sigmoid function.

\subsection{Loss Function and Optimization}
Infection regions occupy a relatively small portion of the image, creating strong foreground--background imbalance. To address this, I use a combined loss:
\begin{equation}
\mathcal{L} = \mathcal{L}_{\text{BCE}} + \mathcal{L}_{\text{Dice}}.
\end{equation}
I train the model with the Adam optimizer (learning rate $10^{-3}$). The best checkpoint is selected based on validation Dice score.

% -------------------------
\section{Experiments and Results}
\subsection{Evaluation Metric}
I evaluate segmentation performance using the Dice coefficient, which measures overlap between prediction $P$ and ground truth $G$:
\begin{equation}
\text{Dice}(P,G) = \frac{2|P \cap G|}{|P| + |G|}.
\end{equation}
Dice is commonly used in medical segmentation because it is sensitive to class imbalance.

\subsection{Main Result}

Two training strategies are investigated in this study. In the first experiment, the model is trained using only COVID-19 images, where infection regions are more consistently present. In the second experiment, the model is trained on the full dataset, including COVID-19, Non-COVID, and Normal images.

Table~\ref{tab:results_compare} summarizes the best validation Dice scores obtained in the two settings.

\begin{table}[ht]
\centering
\caption{Comparison of segmentation results under different training datasets.}
\label{tab:results_compare}
\begin{tabular}{l c}
\toprule
\textbf{Training Dataset} & \textbf{Best Validation Dice} \\
\midrule
COVID-19 only & 0.74 \\
Full dataset (3 classes) & 0.82 \\
\bottomrule
\end{tabular}
\end{table}


% -------------------------
\section{Discussion and Limitations}
The results show that training on the full dataset leads to a higher Dice score compared to using only COVID-19 samples. Although Non-COVID and Normal images often contain empty infection masks, their inclusion helps the model better distinguish true infection regions from normal lung structures. As a result, the model trained on all classes exhibits improved generalization and fewer false positive predictions.


% -------------------------
\section{Conclusion}
This work presents a U-Net-based approach for COVID-19 infection segmentation from chest X-ray images using the COVID-QU-Ex dataset. Experimental results demonstrate that training on the full dataset yields better segmentation performance than training on COVID-19 cases alone. Despite its simplicity, the proposed model achieves reasonable Dice scores and demonstrates the feasibility of infection localization from chest X-ray images.



%  References
\begin{thebibliography}{1}

\bibitem{ronneberger2015unet}
O. Ronneberger, P. Fischer, and T. Brox, ``U-Net: Convolutional Networks for Biomedical Image Segmentation,'' in \emph{MICCAI}, 2015.

\end{thebibliography}

\end{document}
