\documentclass[conference]{IEEEtran}

% -------------------- Packages --------------------
\usepackage{graphicx}
\usepackage{float}
\usepackage{booktabs}
\usepackage{amsmath, amssymb}
\usepackage{siunitx}
\usepackage{hyperref}
\usepackage{xcolor}
\usepackage{subcaption}
\usepackage{cite}

\hypersetup{
  colorlinks=true,
  linkcolor=blue,
  urlcolor=blue,
  citecolor=blue
}

\graphicspath{{figures/}}

% -------------------- Title --------------------
\title{ECG Heartbeat Classification using Gradient Boosting Classifier}

\author{
\IEEEauthorblockN{Ho Huyen Chau -- 23BI14067}
\IEEEauthorblockA{
Machine Learning in Medicine 2026\\
University of Science and Technology of Hanoi (USTH)\\
Email: chauhh.23bi14067@usth.edu.vn
}
}

\begin{document}
\maketitle

% ================= Introduction =================
\section{Introduction}

Electrocardiogram (ECG) signals are widely used in clinical practice. Automatic ECG heartbeat classification has therefore become an important research topic in medical signal processing.

Despite significant progress, the classification of heartbeats on the ECG remains a challenging task. Real-world ECG datasets typically exhibit a strong class imbalance, where normal heartbeats dominate the data, while abnormal events occur much less frequently. 

In this work, I study the problem of ECG heartbeat classification using the MIT-BIH Arrhythmia dataset. Each heartbeat is represented as a fixed-length one-dimensional signal, and the task is formulated as a multi-class classification problem with five heartbeat categories. I focus on a classical machine learning approach based on gradient boosting. This choice allows me to analyze the impact of data imbalance and training strategies while maintaining a relatively simple classification pipeline.

% ================= Dataset =================
\section{Dataset Description}

\subsection{MIT-BIH Arrhythmia Dataset}

The primary dataset used in this study is the MIT-BIH Arrhythmia dataset, originally collected by the Massachusetts Institute of Technology and Beth Israel Hospital and made publicly available via PhysioNet. In this work, I use a preprocessed version of the dataset provided on Kaggle, where individual heartbeats have been segmented and normalized.

Each heartbeat is represented by 187 sampled amplitude values corresponding to approximately 1.5 seconds of ECG recording at a sampling frequency of 125 Hz. Consequently, each sample consists of 188 columns, where the first 187 columns contain the ECG signal values and the last column corresponds to the class label. The dataset is divided into a training set containing 87,554 samples and a test set containing 21,892 samples.

The classification task involves five heartbeat categories: Normal (N), Supraventricular ectopic (S), Ventricular ectopic (V), Fusion (F), and Unknown (Q).

\subsection{Class Distribution and Imbalance}

A major characteristic of the MIT-BIH dataset is its severe class imbalance. Normal heartbeats account for more than 80\% of the samples in both the training and test sets, while some abnormal classes, such as Fusion beats, represent less than 1\% of the data. This imbalance poses a significant challenge for classification models, as high accuracy can be achieved by predicting only the majority class.

\begin{figure}[th]
    \centering
    \includegraphics[width=\linewidth]{mitbih_train_distribution.pdf}
    \caption{Class distribution of the MIT-BIH training set.}
    \label{fig:mitbih_train_dist}
\end{figure}

\begin{figure}[th]
    \centering
    \includegraphics[width=\linewidth]{mitbih_test_distribution.pdf}
    \caption{Class distribution of the MIT-BIH test set.}
    \label{fig:mitbih_test_dist}
\end{figure}

\begin{figure}[th]
    \centering
    \includegraphics[width=\linewidth]{ptb_distribution.pdf}
    \caption{Class distribution of the PTB dataset.}
    \label{fig:ptb_dist}
\end{figure}

\subsection{Visual Exploration of ECG Signals}

To better understand the morphological differences between heartbeat classes, visual inspection of ECG waveforms is performed.

\begin{figure}[th]
    \centering
    \includegraphics[width=\linewidth]{mitbih_examples_per_class.pdf}
    \caption{Example ECG heartbeats for each class in the MIT-BIH dataset.}
    \label{fig:examples_per_class}
\end{figure}

The visual inspection of individual ECG heartbeats reveals noticeable morphological differences among the five classes. Normal beats exhibit a relatively consistent waveform shape, while abnormal classes such as ventricular and supraventricular ectopic beats show irregular patterns. Moreover, fusion and unknown beats display higher variability, reflecting the difficulty of distinguishing these classes.


\begin{figure}[th]
    \centering
    \includegraphics[width=\linewidth]{mitbih_mean_std_per_class.pdf}
    \caption{Mean ECG waveform and $\pm 1$ standard deviation for each heartbeat class.}
    \label{fig:mean_std_per_class}
\end{figure}

Meanwhile, the mean ECG waveforms and their corresponding ±1 standard deviation bands provide further insight into class characteristics. Normal heartbeats show lower variability compared to abnormal classes, while ectopic and fusion beats present wider standard deviation bands. This presents the challenge posed by overlapping patterns between classes.

% ================= Methodology =================
\section{Methodology}

\subsection{Preprocessing}

Each ECG heartbeat in the dataset is represented as a one-dimensional signal consisting of 187 sampled amplitude values. The signals are provided in a preprocessed form, where heartbeats have already been segmented and normalized. No additional filtering or signal transformation is applied in this study. The raw waveform samples are directly used as input features for the classification model.

\subsection{Feature Representation}

No handcrafted features are extracted from the ECG signals. Instead, each heartbeat is treated as a fixed-length feature vector containing the original waveform values. This representation preserves the temporal shape of the ECG signal while keeping the modeling pipeline simple and transparent.

\subsection{Classification Model}

Heartbeat classification is performed using a gradient boosting model based on decision trees. Specifically, the \texttt{HistGradientBoostingClassifier} from the scikit-learn library is employed. The task is formulated as a supervised multi-class classification problem with five output classes corresponding to different heartbeat types.

The gradient boosting algorithm builds an ensemble of decision trees in a sequential manner, where each new tree is trained to correct the errors of the previous ensemble. This method is well suited for tabular data and can model non-linear relationships between ECG waveform values and heartbeat classes.

\subsection{Model Configuration}

The model is trained using the multinomial log-loss objective. The learning rate is set to 0.1, and the maximum number of boosting iterations is set to 300. Tree complexity is controlled by limiting the maximum tree depth to 6, the maximum number of leaf nodes to 31, and enforcing a minimum of 20 samples per leaf node. An L2 regularization term is applied to reduce overfitting.

Early stopping is enabled based on validation loss, using 10\% of the training data as a validation set. Training is stopped if no improvement is observed for 20 consecutive iterations.

\subsection{Evaluation Metrics}

Model performance is evaluated using accuracy and macro-averaged F1 score. Accuracy measures the proportion of correctly classified samples over the entire dataset, while the macro-averaged F1 score computes the F1 score independently for each class and then averages them with equal weight.

Due to the strong class imbalance in the dataset, macro-F1 is considered a more informative metric than accuracy, as it reflects the model’s performance on minority heartbeat classes. Confusion matrices are also used to visualize the final results. 



% ================= Results =================
\section{Results}

\subsection{Validation Results}

Table~\ref{tab:val_results} reports the performance of the gradient boosting classifier on the validation set. The model achieves a high validation accuracy of 95.23\% and a macro-averaged F1 score of 81.25\%, suggesting that it fits the training data well under the validation setting.

\begin{table}[th]
\centering
\caption{Validation performance of the Gradient Boosting classifier.}
\label{tab:val_results}
\begin{tabular}{lcc}
\toprule
Metric & Value \\
\midrule
Accuracy & 0.9523 \\
Macro-F1 score & 0.8125 \\
\bottomrule
\end{tabular}
\end{table}

\subsection{Test Results}

The performance of the final model on the independent test set is summarized in Table~\ref{tab:test_results}. While the overall accuracy remains relatively high at 81.6\%, the macro-averaged F1 score drops significantly to 18.17\%, indicating poor generalization to minority heartbeat classes.

\begin{table}[th]
\centering
\caption{Test performance of the Gradient Boosting classifier.}
\label{tab:test_results}
\begin{tabular}{lcc}
\toprule
Metric & Value \\
\midrule
Accuracy & 0.8160 \\
Macro-F1 score & 0.1817 \\
\bottomrule
\end{tabular}
\end{table}

\subsection{Confusion matrix}

Table~\ref{tab:confusion_matrix} presents the confusion matrix of the Gradient Boosting classifier on the test set. Rows correspond to true labels, while columns correspond to predicted labels.

\begin{table}[th]
\centering
\caption{Confusion matrix on the MIT-BIH test set.}
\label{tab:confusion_matrix}
\resizebox{\linewidth}{!}{
\begin{tabular}{c|ccccc}
\toprule
True / Pred & N & S & V & F & Q \\
\midrule
N (Normal) & 17860 & 258 & 0 & 0 & 0 \\
S (Supra.) & 552 & 4 & 0 & 0 & 0 \\
V (Ventric.) & 1429 & 19 & 0 & 0 & 0 \\
F (Fusion) & 162 & 0 & 0 & 0 & 0 \\
Q (Unknown) & 1608 & 0 & 0 & 0 & 0 \\
\bottomrule
\end{tabular}
}
\end{table}

The confusion matrix shows that the classifier overwhelmingly predicts the Normal class, with almost all minority-class samples misclassified as Normal. This confirms that the model suffers from severe class bias and fails to learn discriminative patterns for rare heartbeat categories.


\begin{thebibliography}{9}

\bibitem{goldberger2000mitbih}
A.~L. Goldberger, L.~A. Amaral, L.~Glass, J.~M. Hausdorff, P.~C. Ivanov,
R.~G. Mark, J.~E. Mietus, G.~B. Moody, C.-K. Peng, and H.~E. Stanley,
``PhysioBank, PhysioToolkit, and PhysioNet: Components of a new research resource for complex physiologic signals,''
\emph{Circulation}, vol.~101, no.~23, pp.~e215--e220, 2000.

\bibitem{shayanfazeli_kaggle}
S.~Fazeli,
``Heartbeat Classification Dataset,''
Kaggle, 2018. Available: \url{https://www.kaggle.com/datasets/shayanfazeli/heartbeat}

\bibitem{ecg_classification_arxiv}
S.~Fazeli,
``ECG Heartbeat Classification: A Deep Learning Approach,''
arXiv preprint arXiv:1805.00794, 2018. Available: \url{https://arxiv.org/abs/1805.00794}

\end{thebibliography}

\end{document}
